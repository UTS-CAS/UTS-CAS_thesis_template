%!TEX TS-program = pdflatex
%!TEX spellcheck = en-AU
%
%% ----------------------------------------------------------------
%% thesis.tex -- MAIN FILE (the one that you compile with LaTeX)
%% ----------------------------------------------------------------

% Set up the document
\documentclass[a4paper, 11pt, oneside]{template/thesis} % Use the "thesis" style, based on the ECS thesis style by Steve Gunn
\graphicspath{{figures/}} % Location of the graphics files (set up for graphics to be in PDF format)
% You can also include multiple paths. eg: \graphicspath{{figures/}{figures/chapter1}{figures/chapter2}}

% Include any extra LaTeX packages required


\hypersetup{urlcolor=blue, colorlinks=false} % Colours hyperlinks in blue, but this can be distracting if there are many links.


% Thesis Preamble 
\thesistitle {Thesis title}
\keywordsvar {Key, Words}
\authors     {Author Name}
\supervisor  {Supervisor Name}
\cosupervisor  {Co-supervisor name}

%% ----------------------------------------------------------------
\begin{document}
\frontmatter   % Begin Roman style (i, ii, iii, iv...) page numbering

% Set up the Title Page

\title {\thesistitlename}


\authors {\texorpdfstring
            {\href{Author@domain.com}{Author Name}}
            {Author Name}
            }
\addresses  {\groupname\\\deptname\\\univname}  % Do not change this here, instead these must be set in the "thesis.cls" file, please look through it instead
\date       {\today}
\subject    {}
\keywords   {\keywordsname}

\maketitle
%% ----------------------------------------------------------------

\setstretch{1.5}  % It is better to have smaller font and larger line spacing than the other way round


% ----------------------------------------------------------------
 %Declaration Page required for the thesis, your institution may give you a different text to place here
\Declaration{

\addtocontents{toc}{\vspace{1em}}  % Add a gap in the Contents, for aesthetics
\vspace{25pt}

I certify that the work in this thesis has not previously been submitted for a degree nor has it been submitted as part of requirements for a degree except as fully acknowledged within the text.

I also certify that the thesis has been written by me. Any help that I have received in my research work and the preparation of the thesis itself has been acknowledged. In addition, I certify that all information sources and literature used are indicated in the thesis.
\vspace{100pt}

Signed:\\
\rule[1em]{25em}{0.5pt}  % This prints a line for the signature

Date:\\
\rule[1em]{25em}{0.5pt}  % This prints a line to write the date
}
\cleardoublepage  % Declaration ended, now start a new page

%% ----------------------------------------------------------------
\fancyhead[RE,LO]{\it Abstract}
% The Abstract Page
\addtotoc{Abstract} % Add the "Abstract" page entry to the Contents
\abstract{
\addtocontents{toc}{\vspace{1em}} % Add a gap in the Contents, for aesthetics
%Suggested by Bruce as para for each: problem(s), method(s), results, conclusion

\lipsum[1]
\todo{TODO}
}

\cleardoublepage % Abstract ended, start a new page
%% ----------------------------------------------------------------
\fancyhead[RE,LO]{\it\leftmark}
\setstretch{1.5} % Reset the line-spacing to 1.3 for body text (if it has changed)

% The Acknowledgements page, for thanking everyone
\acknowledgements{
\addtocontents{toc}{\vspace{1em}} % Add a gap in the Contents, for aesthetics

\lipsum[2]
\todo{TODO}

}
\cleardoublepage % End of the Acknowledgements
% ----------------------------------------------------------------

%\input{./Chapters/Foreword} % Foreword
%\clearpage % End of the Acknowledgements
%\pagestyle{fancy} %The page style headers have been "empty" all this time, now use the "fancy" headers as defined before to bring them back

\fancyhead[RE,LO]{\it\leftmark}
%% ----------------------------------------------------------------
\lhead{\emph{Contents}} % Set the left side page header to "Contents"
\tableofcontents % Write out the Table of Contents

% ----------------------------------------------------------------
\lhead{\emph{List of Figures}} % Set the left side page header to "List if Figures"
\listoffigures % Write out the List of Figures

%% ----------------------------------------------------------------
\listoftables % Write out the List of Tables
%
%%% ----------------------------------------------------------------
\setstretch{1.5} % Set the line spacing to 1.5, this makes the following tables easier to read
\clearpage % Start a new page

%!TEX root = ../thesis.tex
%
\chapter*{Acronyms \& Abbreviations}

\begin{acronym}[AAAAAA]
    \acro{1D}{One-Dimensional }
    \acro{2D}{Two-Dimensional }
    \acro{3D}{Three-Dimensional }
    \acro{CAS}{Centre for Autonomous Systems }
    \acro{UTS}{University of Technology, Sydney  }
\end{acronym}


%% ----------------------------------------------------------------
\cleardoublepage %Start a new page

\listofnomenclature{p{0.93in} p{4.5in}} {

& \textbf{General Notations} \\

$X$ & Robot Pose vector in 2D space. Consists of the position components $x,\ y$ and the orientation component $\phi$. \\


%& \textbf{General Formatting Style}
%\\
%$f (\cdot\cdot\cdot)$ & A scalar valued function
%\\
%$\mathbf{f} (\cdot\cdot\cdot)$ & A vector valued function
%\\
%$[\cdot\cdot\cdot]^T$ & Transpose
%\\
%$| \cdot|$ & Absolute value
%\\
%$\Vert \cdot \Vert$ & Vector length and normalised vector
%\\
%$A \setminus B$ & The set A without any elements that are also in B. Equivalent to $B^c \cap A$.
%\\
%$\mathbb{N}^0$ & Natural numbers greater or equal to zero; e.g. 0, 1, 2, 3, ...
%\\
%$\mathbb{R}^2$ & The set of points in 2D Euclidean space; e.g. (1.2, -57.5).
%\\
%$\mathbb{R}^3$ & The set of points in 3D Euclidean space; e.g. (-3.4, 105.19, 8).
%\\
%\\
%& \textbf{Global Exploration}
%\\
%\\
%& \textbf{Nearest Neighbour Exploration}
%\\
%$\mathcal{A}$ & The robot.
%\\
%$\mathcal{A}(q)$ & The space taken up by the robot in configuration $q$.
%\\
%$\mathcal{C}$ & Configuration space.
%\\
%$\Delta\theta$ & The angle used to determine neighbouring configurations.
%\\
%$\mathcal{H}(q)$ & The information value of a configuration $q$.
%\\
%$\mathcal{P}^{free}_t$ & The set of known free-space at time $t$.
%\\
%$\mathcal{P}^{obs}_t$ & The set of known obstacles at time $t$.
%\\
%$\mathcal{P}^{unk}_t$ & The set of unknown space at time $t$.
%\\
%$Q_a$ & The set of candidate configurations under consideration by AXBAM.
%\\
%$Q_n^t$ & The set of candidate configurations under consideration by NNEA at time $t$.
%\\
%$q$ & A robot configuration, representing a vector of joint angles.
%\\
%$q$ & A robot configuration, representing a vector of joint angles.
%\\
%$q_{curr}^t$ & The robot configuration at time $t$.
%\\
%$q_{nbv}^t$ & The next best view configuration at time $t$.
%\\
%$t$ & A variable representing time.
%\\
%$^oT^{q}_s$ & The transform from map origin to sensor origin based on the robot configuration.
%\\
%$\tau_a$ & The information threshold under which configurations will be discarded in the AXBAM algorithm.
%\\
%$\tau_n$ & The information threshold under which configurations will be discarded in the Nearest Neighbours phase of NNEA.
%\\
%\\
%& \textbf{Frontier Detection}
%\\
%$A_t$ & The active area of the observation at time $t$. 
%\\
%$A_{max}$ & The largest size of any possible active area.
%\\
%$\mathcal{E}_{max}$  & The largest size that any set of cells forming the boundary of any $\mathcal{S}(O_t)$ could be.
%\\
%$\mathcal{E}(O_t)$ & The set of cells in $\mathcal{S}(O_t)$ that are on the edge of the map or adjacent to cells not in $\mathcal{S}(O_t)$.
%\\
%$F_{aa}$ & The set of known frontiers inside the active area under consideration.
%\\
%$F_{new}$ & The set of known frontiers that, when added to the set $F_{t-1}$, will make it become $F_t$.
%\\
%$F_t$ & The set of known frontiers at time $t$.
%\\
%$F_{t-1}$ & The set of known frontiers at time $t-1$.
%\\
%$M$ & The set of cells in the map.
%\\
%$M_{free}$ & The set of free-space cells in the map.
%\\
%$O_t$ & The sensor observation made at time $t$.
%\\
%$o_t$ & The number of individual observations made as part of $O_t$.
%\\
%$\mathcal{S}_{max}$  & The largest size that any set of cells covered by the sensor could be.
%\\
%$\mathcal{S}(O_t)$ & The set of cells covered by the observation $O_t$.
%\\
%\\
}

%% ----------------------------------------------------------------
\cleardoublepage %Start a new page


\listofterms{p{1.1in} p{4.3in}}
{
Autonomous & Without human intervention.
\\
}
  % This file contains all the Acronyms, Abbreveations and Nomenclature.

\begingroup
\clearpage % Start a new page
\lhead{\emph{List of Publications}}
\renewcommand\bibname{List of Publications}
%\input{chapters/pub_list.bbl}
% Before uncommenting this build, chapters/pub_list.tex
\endgroup

%% ----------------------------------------------------------------
\mainmatter % Begin normal, numeric (1,2,3...) page numbering

\setstretch{1.5} % Reset the line-spacing to 1.3 for body text (if it has changed)

%!TEX root = ../thesis.tex
%
% Chapter 1
\chapter{Introduction}
\label{Chapter1}

\lipsum[3-10]

%% Introduction Paragraph

%%%%%%%%%%%%%%%%%%
\section{Background}
\label{Chapter1:Background}

\section{The Acronyms Package}

Acronyms package is pretty neat to manage what you need to expand or not. Using \ac{UTS} will spell it out while using it again the same way like \ac{UTS} will abbreviate it.

To learn how to use the acronyms package, use :

http://tug.ctan.org/macros/latex2e/contrib/acronym/acronym.pdf

%%%%%%%%%%%%%%%%%%
\section{Motivation}
\label{Chapter1:Motivation}


%%%%%%%%%%%%%%%%%%
\section{Scope}
\label{Chapter1:Scope}


%%%%%%%%%%%%%%%%%%
\section{Contributions}
\label{Chapter1:Contributions}


%%%%%%%%%%%%%%%%%%
\section{Publications}
\label{Chapter1:Publications}

\subsection{Directly Related Publications}

\subsection{Related Publications}


%%%%%%%%%%%%%%%%%%
\section{Thesis Outline}
\label{Chapter1:ThesisOutline}



\todo{TODO}
 \clearpage % Introduction

%!TEX root = ../thesis.tex
%
% Chapter 2
\chapter{Review of Related Work}
\label{Chapter2}


\lipsum[11-15]

\todo[color=green]{TODO}
 \clearpage % Review of Related Work

%\input{chapters/chapter3}

%\input{chapters/chapter7} % Conclusion

%% ----------------------------------------------------------------
% Now begin the Appendices, including them as separate files

\addtocontents{toc}{\vspace{2em}} % Add a gap in the Contents, for aesthetics

\addtotoc{Appendices}

\appendix % Cue to tell LaTeX that the following 'chapters' are Appendices

%\input{./Appendices/AppendixA} \clearpage % Data Tables

\addtocontents{toc}{\vspace{2em}} % Add a gap in the Contents, for aesthetics
\backmatter

%% ----------------------------------------------------------------
\label{Bibliography}

\fancyhead[RE,LO]{\emph{Bibliography}} % Change the left side page header to "Bibliography"
\bibliographystyle{template/unsrtnat} % Use the "unsrtnat" BibTeX style for formatting the Bibliography
\bibliography{library} % The references (bibliography) information are stored in the file named "Bibliography.bib"

%compile todo list as well ---------------------------------
%\input{./todo} %To do list


\end{document} % The End
%% ----------------------------------------------------------------
