%!TEX root = ../thesis.tex
%
\chapter*{Acronyms \& Abbreviations}

\begin{acronym}[AAAAAA]
    \acro{1D}{One-Dimensional }
    \acro{2D}{Two-Dimensional }
    \acro{3D}{Three-Dimensional }
    \acro{CAS}{Centre for Autonomous Systems }
    \acro{UTS}{University of Technology, Sydney  }
\end{acronym}


%% ----------------------------------------------------------------
\cleardoublepage %Start a new page

\listofnomenclature{p{0.93in} p{4.5in}} {

& \textbf{General Notations} \\

$X$ & Robot Pose vector in 2D space. Consists of the position components $x,\ y$ and the orientation component $\phi$. \\


%& \textbf{General Formatting Style}
%\\
%$f (\cdot\cdot\cdot)$ & A scalar valued function
%\\
%$\mathbf{f} (\cdot\cdot\cdot)$ & A vector valued function
%\\
%$[\cdot\cdot\cdot]^T$ & Transpose
%\\
%$| \cdot|$ & Absolute value
%\\
%$\Vert \cdot \Vert$ & Vector length and normalised vector
%\\
%$A \setminus B$ & The set A without any elements that are also in B. Equivalent to $B^c \cap A$.
%\\
%$\mathbb{N}^0$ & Natural numbers greater or equal to zero; e.g. 0, 1, 2, 3, ...
%\\
%$\mathbb{R}^2$ & The set of points in 2D Euclidean space; e.g. (1.2, -57.5).
%\\
%$\mathbb{R}^3$ & The set of points in 3D Euclidean space; e.g. (-3.4, 105.19, 8).
%\\
%\\
%& \textbf{Global Exploration}
%\\
%\\
%& \textbf{Nearest Neighbour Exploration}
%\\
%$\mathcal{A}$ & The robot.
%\\
%$\mathcal{A}(q)$ & The space taken up by the robot in configuration $q$.
%\\
%$\mathcal{C}$ & Configuration space.
%\\
%$\Delta\theta$ & The angle used to determine neighbouring configurations.
%\\
%$\mathcal{H}(q)$ & The information value of a configuration $q$.
%\\
%$\mathcal{P}^{free}_t$ & The set of known free-space at time $t$.
%\\
%$\mathcal{P}^{obs}_t$ & The set of known obstacles at time $t$.
%\\
%$\mathcal{P}^{unk}_t$ & The set of unknown space at time $t$.
%\\
%$Q_a$ & The set of candidate configurations under consideration by AXBAM.
%\\
%$Q_n^t$ & The set of candidate configurations under consideration by NNEA at time $t$.
%\\
%$q$ & A robot configuration, representing a vector of joint angles.
%\\
%$q$ & A robot configuration, representing a vector of joint angles.
%\\
%$q_{curr}^t$ & The robot configuration at time $t$.
%\\
%$q_{nbv}^t$ & The next best view configuration at time $t$.
%\\
%$t$ & A variable representing time.
%\\
%$^oT^{q}_s$ & The transform from map origin to sensor origin based on the robot configuration.
%\\
%$\tau_a$ & The information threshold under which configurations will be discarded in the AXBAM algorithm.
%\\
%$\tau_n$ & The information threshold under which configurations will be discarded in the Nearest Neighbours phase of NNEA.
%\\
%\\
%& \textbf{Frontier Detection}
%\\
%$A_t$ & The active area of the observation at time $t$. 
%\\
%$A_{max}$ & The largest size of any possible active area.
%\\
%$\mathcal{E}_{max}$  & The largest size that any set of cells forming the boundary of any $\mathcal{S}(O_t)$ could be.
%\\
%$\mathcal{E}(O_t)$ & The set of cells in $\mathcal{S}(O_t)$ that are on the edge of the map or adjacent to cells not in $\mathcal{S}(O_t)$.
%\\
%$F_{aa}$ & The set of known frontiers inside the active area under consideration.
%\\
%$F_{new}$ & The set of known frontiers that, when added to the set $F_{t-1}$, will make it become $F_t$.
%\\
%$F_t$ & The set of known frontiers at time $t$.
%\\
%$F_{t-1}$ & The set of known frontiers at time $t-1$.
%\\
%$M$ & The set of cells in the map.
%\\
%$M_{free}$ & The set of free-space cells in the map.
%\\
%$O_t$ & The sensor observation made at time $t$.
%\\
%$o_t$ & The number of individual observations made as part of $O_t$.
%\\
%$\mathcal{S}_{max}$  & The largest size that any set of cells covered by the sensor could be.
%\\
%$\mathcal{S}(O_t)$ & The set of cells covered by the observation $O_t$.
%\\
%\\
}

%% ----------------------------------------------------------------
\cleardoublepage %Start a new page


\listofterms{p{1.1in} p{4.3in}}
{
Autonomous & Without human intervention.
\\
}
